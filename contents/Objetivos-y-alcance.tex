\chapter{Objetivos y alcance}

La plataforma integra elementos sofisticados hardware y software que permiten el despliegue de un
robot controlable remotamente por el alumno en una plataforma de juego bajo una interfaz de usuario.

El desarrollo del proyecto supone la aplicación por parte del alumno de conocimientos hardware
(diseño del robot remoto), de comunicaciones (uso de Internet y Bluetooth en el control del robot y
transmisión de imágenes), ingeniería web (desarrollo del cliente e integración en la plataforma
WebLab-Deusto), diseño de interfaces y fomento de la vocación científica (proyecto FECYT Ciencia
Remota).

Los objetivos secundarios del proyecto son los siguientes:
\begin{itemize}
\item Análisis de requisitos, estudio del estado del arte. Documento de requisitos:

Se debe realizar un estudio del arte completo, para conocer cual es el estado actual del mercado en
cuanto a laboratorios remotos robóticos y el fomento de STEM en jóvenes. Además, se realizará un
análisis de requisitos que determine cuales son los requisitos fundamentales del producto final.

\item Revisión de elementos STEM y jóvenes estudiantes:

Se analizarán cuales son los elementos de la formación más influyentes en el área de STEM y se
potenciará su uso mediante la herramienta. Se comprobará el efecto de estos elementos en estudiantes
jóvenes de edades comprendidas entre los 10 y los 18 años.

\item Estudio de la plataforma hardware, Romie:

Se estudiará la plataforma hardware existente (el robot Romie) y se realizarán los cambios
requeridos para adaptarla a las necesidades del proyecto, para poder hacer un correcto uso del
mismo.

\item Control remoto de Romie: interfaz de control:

Se deberá realizar una completa interfaz de control para el robot, que sea capaz de llevar acabo la
comunicación con el mismo desde una interfaz útil e intuitiva para el usuario final.

\item Integración en Weblab-Deusto:

Se aprovechará la actual plataforma de Weblab-Deusto proporcionada por la universidad para el uso de
laboratorios remotos para la comunicación con el robot. Se usará también dicha plataforma para
controlar los accesos, concursos y estadísticas, así como para la gestión de colas y autenticación
de los usuarios.

\item Diseño de la plataforma de juego: puntuación, registro, integración del robot:

Se creará una plataforma de juego basada en un juego tipo trivial, en el que el usuario deberá
responder preguntas para obtener una mejor puntuación, y así poder participar en un concurso. Se
deberá analizar cuidadosamente los valores de las preguntas, las puntuaciones, los rankings y el
registro de los usuarios.

\item Integración en la plataforma de juego de un escenario de indagación como lucha frente a la
pseudociencia. Trabajo con un equipo de psicólogos:

El laboratorio de psicología de la Universidad de Deusto, Labpsico, proporcionará un experimento
sobre la lucha contra la pseudociencia que se incluirá en el juego en una de sus modalidades. Los
datos relevantes a la investigación psicológica serán enviados a Labpsico, mientras que el usuario
recibirá un bonus en el juego según su buena realización de la actividad de psicología.

Entorno de programación de Romie basado en Blockly o Scratch:
Se creará un entorno de programación visual para el robot, basado en Google Blockly o en Scratch,
todavía por decidir, dependiendo de la investigación previa. Este entorno servirá para acercar a los
jóvenes la programación.

\item Diseminación de la plataforma: Despliegue en el Proyecto FECYT “Ciencia Remota” y prueba de la
plataforma por al menos 500 alumnos:

El producto se desplegará en el proyecto del FECYT “Ciencia Remota” en el que varias instituciones
trabajan por llevar la experimentación remota a más lugares. Para ello, y como demostración del
potencial del proyecto, se harán varias pruebas públicas del producto en las que participarán en
total no menos de 500 usuarios finales de diversos colegios e instituciones.

\item Elaboración de un informe de usabilidad:

Se realizará un completo informe de la usabilidad de la interfaz de usuario, para comprobar que es
intuitiva y fácil de usar. Para ello se realizarán técnicas heurísticas complejas y se usarán varios
métodos a nuestro alcance.

\item Sistema de seguimiento y puntuación general

\end{itemize}
