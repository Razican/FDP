\chapter{State of the Art}

This project aims to create a serious game based on a robot in a remote laboratory. Moreover, it
aims to create a complete visual programming experience to teach the basics of programming. Thus,
in this state of the art, I will cover the three main topics that the project uses as a base and I
will deepen into them.

\section{Remote laboratories}

As we have seen in previous sections, remote laboratories can be really helpful in teaching main
concepts about science and technology. In this section I will analyse which is the current status of
remote experimentation around the world and I will show examples of how do remote laboratories work
in various scenarios.

\subsection{Global Online Laboratory Consortium (GOLC)}

The Global Online Laboratory Consortium or \textit{GOLC} is an organization that focuses on
the promotion of the development of remote laboratories for educational use. They commonly promote
remote laboratories through conferences~\cite{golc1st}. They also support and encourage the sharing
of these laboratories between institutions.

For promoting laboratories, they created an award (Figure~\ref{fig:golc_award}) for remote
experimentation and another one for simulated experimentation.

\begin{figure}[h]
	\centering
	\includegraphics[width=.4\textwidth]{fig/golc_award}
	\caption{GOLC Online Laboratory Award 2015}\label{fig:golc_award}
\end{figure}

\subsection{WebLab-Deusto}

WebLab-Deusto is a remote laboratory located at the University of Deusto, Bilbao. There are multiple
types of laboratories there, and all is being controlled by a software they developed, called
WebLab. Moreover, Pablo Orduña, one of it's main researches developed a complete federation model
to be able to share laboratories across the world~\cite{porduna_phd}.

% TODO insert picture from WebLab

In Weblab-Deusto they created one of the most used remote laboratories in electronics teaching. It's
called VISIR~\cite{visir}, and it recreates circuits made visually by students in real hardware so
that students can take real measurements.

\begin{figure}[h]
	\centering
	\includegraphics[width=.4\textwidth]{fig/weblab}
	\caption{WebLab-Deusto logo.}
\end{figure}

\subsection{Robotic remote experiment}

Some of the laboratories listed before have currently implementations of robotic experiments in
as a teaching material in some areas. There are

% TODO iRobot rig, Romie, WebLab-Bot

\section{Simulation}

Even if the project will not be based on simulated robotics but in real robots, currently many
many projects use simulations to bring some of the experience to users.

\section{Serious Games}

\section{Visual Programming}

\subsection{Scratch}

\subsection{Blockly}
