\chapter{Objectives and Scope}

The platform will integrate sophisticate hardware and software elements that allow us to deploy a
remote controlled robot so that it can be used by students in a game platform using the provided
user interface.

The development of the project will require knowledge about the hardware (for the maintenance and
modifications of the robot), knowledge about communication protocols (\acrshort{http},
\acrshort{tcp}/\acrshort{ip} and Bluetooth) for command and image transmission, knowledge about web
engineering (for the develoment of the client and the interaction with the WebLab-Deusto platform)
and user interface design knowledge for creating a user-friendly interface taking into account the
requirements asked by \acrshort{fecyt} Remote Science.

The secondary objectives of the project will be the following:
\begin{itemize}
\item \textbf{Requirement analysis, state of the art study. Requirements specification}:

A complete state of the art study will be performed to know th current market situation regarding to
remote robotic laboratories and the \acrshort{stem} promotion in young people. Moreover, a
requirement analysis will be performed which will give us the final requirement specification.

\item \textbf{\acrshort{stem} element revision for young students}:

We will analyze which are the most influential elements in the education of students in the
\acrshort{stem} area so that they can be maximized when developing the platform. They will be proved
with young students aged between 10 and 18 years old.

\item \textbf{Current hardware platform analysis and modification}:

We will study the possibilities of the current hardware in WebLab-Deusto (the robot Romie) and the
needed changes will be developed to adapt it to the needs of the project.

\item \textbf{\acrshort{api} for remote control}):

A complete control \acrshort{api} will be developed, able of communicating with the robot with a
a simple interface for the client software taking into account the restrictions of the WebLab-Deusto
environment.

\item \textbf{Trivial type game platform development - registration, game design, score and robot
control}:

A game platform will be created based on a simple trivial game, where the user will have to answer
the proposed questions to obtain a high score. A contest will take place where these users will get
a prize. Game rules must be carefully analyzed so that the game will not be too easy nor too
difficult.

\item \textbf{Integration of a psychological experiment for fighting against pseudoscience. Work
with a psychologist group}:

The psychology laboratory of the University of Deusto, Labpsico will provide an experiment about the
fight against pseudoscience that will be added to the game in one of its game modes. The relevant
data for the psychological research will be sent to Labpsico, while the user will receive a bonus in
the game depending on his or her performance in the psychological activity.

\item \textbf{Visual programming environment for the robot}:

A visual programming environment for the robot will be created based on one of the most known
platforms: Blockly or Scratch, still to be decided, depending on the previous research. This
environment will be used to teach the basics of programming to young students.

\item \textbf{Integration in WebLab-Deusto}:

We will use the WebLab-Deusto platform provided by the University of Deusto so that we can deploy
the experiment in a production environment along with the rest of the experiments. This provides the
experiment with a simple interface for the communication with the experiment server and with the
robot. It will also be in charge of managing user queues and user authentication.

\item \textbf{Platform dissemination: Deployment in the ``Ciencia Remota'' project of
\acrshort{fecyt} and testing by students}:

The game will be deployed in th ``Ciencia Remota'' projct of \acrshort{fecyt} where many
institutions work to bring remote experimentation to more places. For that, and as a demonstration
of the potential of the project, some public test will be performed where students from various
schools will take part.

\item \textbf{Usage statistics report}:

A complete report will be generated to learn from the use statistics. This will provide us with
information on how to improve the game.

\end{itemize}
