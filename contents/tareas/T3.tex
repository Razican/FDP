\subsection{Tarea 3}

{
\fontfamily{cmss}
\setlength{\extrarowheight}{4pt}
\begin{center}
	\begin{tabular}{!{\VRule[4pt]}p{200pt}!{\VRule[2pt]}p{100pt}!{\VRule[4pt]}}
		\specialrule{4pt}{0pt}{0pt}
		\multicolumn{2}{!{\VRule[4pt]}c!{\VRule[4pt]}}{\Large{\textbf{\MakeUppercase{Hoja de tareas}}}} \\
		\specialrule{2pt}{0pt}{0pt}
		\multicolumn{2}{!{\VRule[4pt]}l!{\VRule[4pt]}}{\textbf{Nombre:} Iban Eguia Moraza} \\
		\multicolumn{2}{!{\VRule[4pt]}l!{\VRule[4pt]}}{\textbf{Fecha:} 29 de marzo de 2015} \\
		\specialrule{2pt}{0pt}{0pt}
		\multirowcell{2}{\textbf{Identificación de Tarea : T3}\\
		\textbf{Descripción:}\\
		Se debe diseñar la lógica interna del robot y de la plataforma de control del mismo, para
		hacerla accesible, simple y estable.}
		                                                      & \makecell{ \\[-1.5ex]\textbf{Duración : 8 días}\vskip0.5ex} \\
		\Xcline{2-2}{2pt}

		                                                      & \makecell{ \\[-0.5ex]\textbf{Esfuerzo: 32 horas}\vskip0.5ex} \\
		\Xcline{2-2}{2pt}

		                                                      & \multirowcell{2}{\textbf{Tareas previas:} \\
		                                                      	T1} \\
		\Xcline{1-1}{2pt}

		\textbf{Criterios de terminación:} & \\
		El robot será controlable mediante comandos. El comité de dirección será el encargado de
		validar y aceptar la tarea.
		                                                      & \\[-3ex]
		\Xcline{2-2}{2pt}
		                                                      & \multirowcell{2}{\textbf{Recursos:}\\
		                                                      	Programador} \\
		\Xcline{1-1}{2pt}
		\textbf{Competencias, conocimientos y notas:} & \\

		{El programador deberá conocer el funcionamiento del robot, así como la estuctura del
		laberinto. Además, deberá ser capaz de programarlo y de solucionar problemas que pudieran
		ocurrir.} & \\
		\specialrule{4pt}{0pt}{0pt}
	\end{tabular}
\end{center}
}

\clearpage
