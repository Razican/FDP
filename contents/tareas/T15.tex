\subsection{Tarea 15}

{
\fontfamily{cmss}
\setlength{\extrarowheight}{4pt}
\begin{center}
	\begin{tabular}{!{\VRule[4pt]}p{200pt}!{\VRule[2pt]}p{100pt}!{\VRule[4pt]}}
		\specialrule{4pt}{0pt}{0pt}
		\multicolumn{2}{!{\VRule[4pt]}c!{\VRule[4pt]}}{\Large{\textbf{\MakeUppercase{Hoja de tareas}}}} \\
		\specialrule{2pt}{0pt}{0pt}
		\multicolumn{2}{!{\VRule[4pt]}l!{\VRule[4pt]}}{\textbf{Nombre:} Iban Eguia Moraza} \\
		\multicolumn{2}{!{\VRule[4pt]}l!{\VRule[4pt]}}{\textbf{Fecha:} 29 de marzo de 2015} \\
		\specialrule{2pt}{0pt}{0pt}
		\multirowcell{2}{\textbf{Identificación de Tarea : T15}\\
		\textbf{Descripción:}\\
		Se desarrollará una interfaz sencilla de usar y que permita acceder a todas las funciones de
		esta modalidad de juego.}
		                                                      & \makecell{ \\[-1.5ex]\textbf{Duración : 4 días}\vskip0.5ex} \\
		\Xcline{2-2}{2pt}

		                                                      & \makecell{ \\[-0.5ex]\textbf{Esfuerzo: 16 horas}\vskip0.5ex} \\
		\Xcline{2-2}{2pt}

		                                                      & \multirowcell{2}{\textbf{Tareas previas:} \\
		                                                      	T12} \\
		\Xcline{1-1}{2pt}

		\textbf{Criterios de terminación:} & \\
		Los usuarios podrán programar el robot visualmente para hacerle cumplir los retos
		necesarios. El comité de dirección será el encargado de validar y aceptar la tarea.
		                                                      & \\[-3ex]
		\Xcline{2-2}{2pt}
		                                                      & \multirowcell{2}{\textbf{Recursos:}\\
		                                                      	Diseñador} \\
		\Xcline{1-1}{2pt}
		\textbf{Competencias, conocimientos y notas:} & \\

		{El diseñador deberá conocer el funcionamiento de las APIs de la programación visual, y
		deberá saber cómo integrarlas en una página web.} & \\
		\specialrule{4pt}{0pt}{0pt}
	\end{tabular}
\end{center}
}

\clearpage
