\subsection{Tarea 10}

{
\fontfamily{cmss}
\setlength{\extrarowheight}{4pt}
\begin{center}
	\begin{tabular}{!{\VRule[4pt]}p{200pt}!{\VRule[2pt]}p{100pt}!{\VRule[4pt]}}
		\specialrule{4pt}{0pt}{0pt}
		\multicolumn{2}{!{\VRule[4pt]}c!{\VRule[4pt]}}{\Large{\textbf{\MakeUppercase{Hoja de tareas}}}} \\
		\specialrule{2pt}{0pt}{0pt}
		\multicolumn{2}{!{\VRule[4pt]}l!{\VRule[4pt]}}{\textbf{Nombre:} Iban Eguia Moraza} \\
		\multicolumn{2}{!{\VRule[4pt]}l!{\VRule[4pt]}}{\textbf{Fecha:} 29 de marzo de 2015} \\
		\specialrule{2pt}{0pt}{0pt}
		\multirowcell{2}{\textbf{Identificación de Tarea : T10}\\
		\textbf{Descripción:}\\
		Se integrará el experimento con la plataforma existente de Romie, haciendo uso de la
		interfaz creada para la primera modalidad de juego.}
		                                                      & \makecell{ \\[-1.5ex]\textbf{Duración : 8 días}\vskip0.5ex} \\
		\Xcline{2-2}{2pt}

		                                                      & \makecell{ \\[-0.5ex]\textbf{Esfuerzo: 32 horas}\vskip0.5ex} \\
		\Xcline{2-2}{2pt}

		                                                      & \multirowcell{2}{\textbf{Tareas previas:} \\
		                                                      	T9} \\
		\Xcline{1-1}{2pt}

		\textbf{Criterios de terminación:} & \\
		Se podrá realizar el experimento psicológico directamente desde la primera modalidad de
		juego, recibiendo un bonus de puntuación para el trivial. El comité de dirección será el
		encargado de validar y aceptar la tarea.
		                                                      & \\[-3ex]
		\Xcline{2-2}{2pt}
		                                                      & \multirowcell{2}{\textbf{Recursos:}\\
		                                                      	Programador - 80\% \\
		                                                      	Diseñador - 20\%} \\
		\Xcline{1-1}{2pt}
		\textbf{Competencias, conocimientos y notas:} & \\

		{El programador y el diseñador deberán conocer técnicamente tanto el experimento como la
		primera modalidad de juego para poder integrarlos en una sola aplicación.} & \\
		\specialrule{4pt}{0pt}{0pt}
	\end{tabular}
\end{center}
}

\clearpage
