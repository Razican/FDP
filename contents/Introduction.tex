\chapter{Introduction}

This project aims to create a complete game experiment using a remote laboratory. At our best of our
knowledge, there is no game that uses remote laboratories as a platform to provide a resource in the
game, so this project aims to create the first game experience ever in a remote laboratory platform.
Moreover, it has been sponsored by the \acrshort{fecyt} (\acrlong{fecyt}) foundation's ``Ciencia
remota'' project, where this project will be deployed in museums.

We will have to develop the game using the WebLab-Deusto platform, using their hardware and their
software libraries, and we will have to integrate the game with the robot Romie in its labyrinth.
For that, we will have to learn and use many technologies, learn to use the WebLab-Deusto's network
and create a game that could be used by dozens of people.

Taking that into account, we can see how this project will confront some big challenges:

\begin{itemize}

\item Integrate hardware elements with software interface to create a game that will be used by
young students to play an learn.

\item Use the available resources of a platform never before used as a game platform: remote
laboratories.

\item Design and develop a game that would use real hardware for the gaming experience available
at all times.

\item Demonstrate the viability of the game and the reliability of the platform in real events with
users playing.

\item Learn and use the tools and libraries provided by WebLab-Deusto.

\end{itemize}

Furthermore, this project does not only try to create a game platform. Together with the psychology
laboratory of the University of Deusto, Labpsico~\cite{labpsico_web}, we will develop a complete
experience to fight against pseudoscience and gather valuable data from the experiment. We will use
the same game platform to integrate the psychological experiment they will provide us so that users
completing the psychological experience will receive better starting scores in the game.

This task in itself carries many challenges that we will have to solve:

\begin{itemize}

\item Work in an interdisciplinary environment with psychologists to design the software
integration.

\item Integrate a complete psychology experiment in a game using remote laboratories.

\end{itemize}

This challenges will need good group work skills as well as good communication skills, since we will
need to understand each others to develop the best possible solution and integrate it with the
current University of Deusto's remote laboratory platform, WebLab-Deusto.

Moreover, and since WebLab-Deusto~\cite{weblab_web} is part of DeustoTech
Learning~\cite{dtlearning_web}, the department of DeustoTech~\cite{deustotech_web} that creates
learning tools, we decided that we should create a learning experience based on the robot. This will
be one of the key parts of the project, since we do not currently have an experiment for young
students to program robots in WebLab-Deusto.

That will be accomplished using a visual programming environment that will be integrated in the
WebLab-Deusto platform where students will be able to program the robot using blocks with the latest
technologies. All this visual programming environment will be usable from the web interface of
WebLab-Deusto, with its queue and priority management.

Of course, this new scenario proposes more challenges that will have to be solved:

\begin{itemize}

\item Research about visual programming technologies and decide which to use.

\item Learn and use a new technology to create a visual programming environment.

\item Connect the environment with the robot, taking into account that it cannot break previous
developments.

\item Control the code execution to avoid security issues and acknowledge the user in eventualities.

\end{itemize}

These three scenarios will be the ones developed in this project. They will show many of the learned
abilities in the university and they will require further learning to be able to develop the
project.

In this document this project is presented. We will first take a look to the background of the
project, where we will see what other technologies and platforms exist in the world, and the
rationale for the project.

We will then define the objectives and the scope of the project, where we will define the project
perfectly and explain how the project will be developed. Finally we will show the organizational
structure of WebLab-Deusto, where the project will be developed.

After that, we will do the planning of the project, where we will be able to see the development
dates and workloads, as well as the Gantt and precedence diagrams. Then we will have the budget of
the project.

After that, we will start with the development section, where we will first analyze the technologies
we will use for this project. We will go through all the different aspects of the software, from the
embedded computers, servers and all the software technologies needed for the development of the
client and server sides. We will also research on the hardware technology used.

Once we have all the planning and technologies defined, we will start with the development of the
project itself. This section will be divided into three subsections, one for each of the scenarios
of the project. The first of them will show the game development, the second one will show the
Labpsico experiment integration and the last one will show the visual programming interface
development. Each of them will explain the requirements, the design and the deployment, along with
the testing and issue management.

Finally, we will explain the conclusions and results of this project, and we will think on some
future developments. We will show the data of all the uses of the game to show the viability of the
project.
