\chapter{Introduction}

This project aims to create a complete game experiment using a remote laboratory. After the research
that has been conducted for this project, there has been no sign of other game that uses remote
laboratories as a platform to provide a resource in the game, so this project aims to create the
first game experience ever in a remote laboratory platform. Moreover, it has been sponsored by the
\acrshort{fecyt} (\acrlong{fecyt}) foundation's ``Ciencia remota'' project, where this project will
be deployed in museums.

The game will have to be developed using the WebLab-Deusto platform, using their hardware and their
software libraries, and the game will have to be integrated with the robot \emph{Romie} in its
labyrinth. For that, many technologies will have to be learned an used, WebLab-Deusto's network will
have to be understood and a game that could be used by dozens of people will have to be created.

Taking that into account, it can be understood how this project will confront some big challenges:

\begin{itemize}

	\item Integrate hardware elements with software interface to create a game that will be used by
	young students to play and learn.

	\item Use the available resources of a platform never before used as a game platform (remote
	laboratories) to create a complete game experience.

	\item Design and develop a game that will use real hardware for the gaming experience available
	anytime and anywhere.

	\item Demonstrate the viability of the game and the reliability of the platform in real events
	with real users playing.

	\item Learn and use the tools and libraries provided by WebLab-Deusto.

\end{itemize}

Furthermore, this project does not only try to create a game platform. Together with the psychology
laboratory of the University of Deusto, Labpsico~\cite{labpsico_web}, a complete experience to fight
against pseudoscience will be integrated, and valuable data will be gathered from the experiment.
The same game platform will be used to integrate the psychological experiment Labpsico will provide
so that users completing the psychological experience will receive better starting scores in the
game.

This task in itself carries more challenges that will have to be solved:

\begin{itemize}

	\item Work in an interdisciplinary environment with psychologists to design the software
	integration.

	\item Integrate a complete psychology experiment in a game using remote laboratories.

\end{itemize}

This challenges will need good group work skills as well as good communication skills, since
complete understanding will be needed between the teams to develop the best possible solution and
integrate it with the current University of Deusto's remote laboratory platform, WebLab-Deusto.

Moreover, and since WebLab-Deusto~\cite{weblab_web} is part of DeustoTech
Learning~\cite{dtlearning_web}, the department of DeustoTech~\cite{deustotech_web} that creates
learning tools, it has been decided that a learning experience should be created based on the robot.
This will be one of the key parts of the project, since there is currently no experiment for young
students to program robots in WebLab-Deusto.

That will be accomplished using a visual programming environment that will be integrated in the
WebLab-Deusto platform where students will be able to program the robot using blocks with the latest
technologies. All this visual programming environment will be usable from the web interface of
WebLab-Deusto, with its queue and priority management.

Of course, this new scenario proposes more challenges that will have to be solved:

\begin{itemize}
	\item Research about visual programming technologies and decide which to use.

	\item Learn and use a new technology to create a visual programming environment.

	\item Connect the environment with the robot, taking into account that it cannot break previous
	developments.

	\item Control the code execution to avoid security issues and acknowledge the user in
	eventualities.
\end{itemize}

These two scenarios, along with the psychological experiment will be the ones developed in this
project. They will show many of the learned abilities in the university and they will require
further learning to be able to develop the project.

In this document this project is presented. First, a look to the background of the project will be
taken, where other technologies and platforms that exist in the world will be seen, and the
rationale for the project will be explained.

The objectives and the scope of the project will then be defined, where the project will be
completely defined and its development will be explained. Finally the organizational structure of
WebLab-Deusto will be shown, where the project will be developed. Then, the project's organization
will be explained with both WebLab-Deusto / DeustoTech Learning and Labpsico.

After that, the planning of the project will be analyzed, where the development dates and workloads
for the project will be stated, as well as the Gantt and precedence diagrams. Then the budget for
the project will be analyzed.

After the planning, the development of the project itself will be explained. This section will be
divided into two subsections. The first of them will show the game development, and after that
explanation, the psychological experiment integration will be explained. The second one will show
the visual programming interface development. Each of them will explain the requirements, the design
and the deployment, along with the testing and issue management. Both of them will have a small user
manual to explain how to use them.

Once the project has been explained, the used technologies will be analyzed. The analysis will go
through all the different aspects of the software, from the embedded computers and server software
to all the software technologies needed for the development of the client and server sides. The
hardware technology used will also be researched.

Finally, the conclusions and results of this project will be presented and some possible future
developments will be appointed. The usage data will be shown and analyzed to show the viability and
reliability of the project.
