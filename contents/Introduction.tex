\chapter{Introduction}

\section{Background}

% TODO: insert images

\subsection{Remote Laboratories}

Remote laboratories are usual laboratories, with the peculiarity that they can be accessed through
the Internet \cite{remote_labs}. They give students the option to use the laboratory even from home
or while the university or school is closed. This means that the students are able to do their
homework or experimentation using the equipment at their school or university without the need for
them to be physically there.

Currently they are becoming more popular due to the competitive advantage they can give to schools
and universities. Since there is no need for the student to be physically in the laboratory, the
laboratory does not need to be physically in the school or university, giving the option to share
laboratories between institutions and thus giving important economic benefits without reducing
the practice time of the students, or even increasing it.

\subsection{Serious Games}

Serious games are video games that do not only entertain, but they manage to teach. Thanks to that,
they can be used to improve the quality of the learning environment for students. Moreover, since
games in many cases attract better the attention of young people, they can even be a better tool for
teaching, at least, the basic concepts of some subjects.

\subsection{WebLab-Deusto}

WebLab-Deusto is a remote laboratory facility located in the University of Deusto \cite{weblab},
Bilbao. Since 2001, it has been providing students with remote laboratories to complete their
academic learnings. Since then, it has been extended and many of it's laboratories is available
all over the world. It's no longer a laboratory only made for microelectronics university students
since nowadays it serves schools and universities everywhere to provide them with remote
laboratories.

Among others, it serves an experiment to prove and measure the Archimedes' principle, an experiment
to program and control a robot and some microelectronics experiments with PLDs and FPGAs. Moreover,
there are more laboratories in development, such as an elevator to teach students how to program and
control them, an experiment with an aquarium where students can give them food and, of course, the
project presented here: a complete learning experience given by a robot.

\section{Motivation}

Technology is changing every day, and among other luxuries, it gives us the ability to improve the
way we perform our tasks. Nowadays, one of the most important things in our lives, that consumes
more than a quarter of it is our education. In a strong belief that technology can contribute to
give future generations better tools for learning, serious games seem to be a viable option I would
like to deepen. Having the opportunity to do so in a remote environment such as Weblab-Deusto gives
us the ability to create a remote laboratory, with the potential for being used by many people
around the world, and thus make a difference in learning environments.
