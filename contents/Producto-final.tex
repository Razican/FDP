\chapter{Producto final}

El producto final se dividirá en dos modalidades de juego relacionadas con un robot. La primera,
será un juego de tipo trivial en el que los usuarios recibirán una puntuación dependiendo de las
preguntas acertadas, que estarán situadas en lugares concretos del laberinto. Además, y gracias a la
integración con Weblab-Deusto, existirá la posibilidad de habilitar una actividad de experimentación
psicológica antes de comenzar el juego en sí, que permitirá hacer una experimentación muy útil para
Labpsico y apoyará la lucha contra la pseudociencia mediante un experimento proporcionado por ellos.

La segunda experiencia será un entorno de desarrollo visual para el robot, en el que el usuario
podrá crear programas simples y hacerlos funcionar en el robot. Podrá controlar los aspectos
fundamentales, y esto le permitirá resolver problemas simples. Será muy útil para acercar el mundo
de la programación a jóvenes. Para ello, el usuario dispondrá de módulos de instrucciones de
programación que podrá combinar a su antojo.
