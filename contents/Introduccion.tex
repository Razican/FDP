\chapter{Introducción}

\section{Antecedentes}

En respuesta a la fuerte y creciente demanda de nuevas herramientas de aprendizaje, se presenta una
nueva experiencia de aprendizaje, basada en laboratorios remotos~\cite{remote_labs}. Este proyecto
trata de crear una plataforma de juego para jóvenes de entre 10 y 18 años para introducirlos a la
ciencia, tecnología, ingeniería y las matemáticas de una manera amena.

La plataforma de juego se dividirá en dos escenarios principales. El primero de ellos será un juego
de tipo trivial en el que el usuario podrá controlar el robot en un laberinto y responder algunas
preguntas que darán puntos al usuario. Los ganadores recibirán un premio. Además, en este escenario,
se hará un experimento psicológico con intención de ayudar en la lucha contra la pseudociencia.

El segundo escenario será un servicio en el que el usuario podrá programar el robot desde una
perspectiva visual. Después de eso, el robot deberá realizar algunas tareas simples en el mismo
laberinto del primer escenario.

En este documento se presenta este proyecto y se propone una implementación del mismo. Para ello, se
definirán los objetivos y el alcance del proyecto, la descripción de la realización y las
condiciones de ejecución del mismo. El formato del documento se ha basado en la guía de formato de
memorias~\cite{formato}.

\section{Motivación}

La tecnología está cambiando cada día, y entre otros lujos, nos da la posibilidad de mejorar la manera
en la que realizamos nuestras tareas. Hoy en día, una de las cosas más importantes en nuestras
vidas, que consume más de un cuarto de ella es nuestra educación. Con una fuerte convicción de que
la tecnología puede contribuir a dar a las futuras generaciones mejores herramientas para el
aprendizaje, los serious games parecen una opción viable que me gustaría profundizar. Teniendo la
oportunidad de hacerlo en un entorno remoto como WebLab-Deusto~\cite{weblab} nos da la posibilidad
de crear un laoratorio remoto, con el potencial de ser usado por muchas personas alrededor del
mundo, y así marcar la diferencia en entornos de aprendizaje.
