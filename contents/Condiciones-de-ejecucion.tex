\chapter{Condiciones de ejecución}

El entorno de trabajo se dividirá entre el laboratorio Weblab-Deusto, en el que se realizarán las
tareas necesarias de mantenimiento e implementación de hardware y software y las oficinas de
DeustoTech, en la sección de Learning. Se harán 4 horas de trabajo diarias, pudiendo exigirse horas
extra en el caso de eventos en los que se llevará la plataforma de juegos a pruebas, como por
ejemplo, ForoTech. Durante el desarrollo, el comité de dirección será el responsable del proyecto.

Se usarán los medios provistos por la Universidad de Deusto, y además se hará uso del material
personal del equipo de desarrollo, que en este caso será un portátil y una licencia software de
Sublime Text 3.

El control de cambios se llevará a cabo por petición del equipo de Weblab-Deusto o Labpsico,
dependiendo del caso. Se realizará una valoración de las posibles repercusiones del cambio para el
plan de trabajo, y se tomará una decisión que deberá ser ratificada por el comité de dirección.

Para la recepción de productos, el equipo de desarrollo presentará el producto a la dirección de
Weblab-Deusto y, de ser necesario, a Labpsico. Los equipos de Weblab-Deusto y Labpsico valorarán el
producto y realizarán las pruebas necesarias para su aceptación. La decisión final deberá ser
comunicada al equipo de desarrollo en un plazo no mayor a 3 días, en caso de que se supere dicho
plazo, se considerará aceptado.
