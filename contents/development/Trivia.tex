\section{Trivia Game}

The first modality of the game is a trivia-type game, that will use Romie, the robot in
WebLab-Deusto and its labyrinth to create an experience for learning general knowledge. Moreover,
since the trivia questions will be configurable, so that the game can be adapted.

\subsection{Software and Hardware Requirements}

This software will have tight requirements in terms of user-friendliness, communication stability
and security, since it must be developed using as far as possible current hardware and be easy to
use by young students. Moreover, since it will be presented in crowded events, it must support high
load and availability.

Furthermore, due to the requirements of the project, it must be integrated with the WebLab-Deusto
platform and must be deployable in that environment. Thus, the requirements specification will be
as it follows:

\subsubsection{Software requirements:}

\begin{itemize}
	\item The software must be able to communicate with the robot.
	\item The software must be able to send control commands to the robot.
	\item The software must be able to receive command results from the robot.
	\item The software must be integrated in the WebLab-Deusto platform as a new experiment.
	\item The software must have an easy to use user interface based on human-computer interaction
	principles.
	\item The software must be stable enough to support tens of accesses per hour.
	\item The software must provide enough questions so that the user never finishes with them and
	can be randomly selected.
	\item The game must increase difficulty as the user gets more points.
	\item The game must finish in less than 15 minutes.
	\item A ranking must be provided after finishing the game for the user to know its ranking.
\end{itemize}

\subsubsection{Hardware requirements:}

\begin{itemize}
	\item The hardware must be placed in WebLab-Deusto.
	\item The robot must never get blocked, so in the case of an incident it must be automatically
	recovered.
	\item The cameras must be accessible from the Internet.
	\item The robot must be controlled via Bluetooth.
	\item The robot must never run out of power.
	\item The robot must be able to read all the \acrshort{rfid} tags with at least 99~\% accuracy.
	\item The robot must use the current labyrinth in WebLab-Deusto.
	\item No new hardware can be added to the current WebLab server (Plunder).
\end{itemize}

\subsection{Design Specification}

Taking into account the previous requirements, it has been decided to do a small hardware redesign
and a complete software design for the project. We will now see the hardware and software design
specifications.

\subsubsection{Hardware Specification}

Current robot is deployed with a simple \acrshort{rfid} reader (model ID-12) not capable of reading
further than 120mm~\cite{rfid}, which generates some reading inconsistencies from the distance the
sensor is located in the robot (about 100-120mm from the ground). For that reason, it has been
decided to use a new module, the new module will be model ID-20LA. This will give the robot a much
higher reliability when reading \acrshort{rfid} tags, since the range of the new sensor is
180mm~\cite{rfid}.

On the other hand, there is currently an issue with the availability of the robot. It is powered
with a 2Ah \acrshort{lipo} battery, and is recharged when needed. This has a big issue, since as we
have seen, in high load conditions would not meet the required availability, and furthermore, in
weekends or holidays, we will not be able to change and recharge the battery, so it has been decided
to deploy a cable installation from the roof of the ceiling of the laboratory, and the design of the
robot has been adapted so that the cables do not get stuck in the labyrinth.

The rest of the robot will be used as it is, since it provides with the needed capabilities for the
needs of the project: It has a wall sensor capable of avoiding crashes with walls, infrared sensors
to detect the lines in the ground, motors and wheels capable of moving the robot, Bluetooth
connection to communicate with it and Arduino microcontroller, to install the needed firmware. It
also provides

\subsubsection{Software Specification}

The software in for this implementation will be divided modularly, thinking on scalability and
code reuse. The application must be built on top of WebLab-Deusto, so we will use as much as
possible the provided \acrshort{api}s. Moreover, and due to deployment needs, the software will be
divided between the robot, an intermediate server and WebLab software.

The robot will use a slightly modified version of the current firmware, since we need it not to get
blocked by any wall in case of crash and we need a more reliable implementation. Nevertheless, the
external Bluetooth \acrshort{api} will be the same as the one that was already implemented at the
beginning of the project.

The intermediate server will provide a small \acrshort{rest} \acrshort{api} in a small Python
server. Its only duty will be to provide a simple interface for the robot using \acrshort{http}
instead of Bluetooth, needed due to deployment constraints.

Then, the experiment server required by the WebLab-Deusto architecture, will provide a WebLab
command \acrshort{api}, that will be callable by the client of the experiment. This server will be
developed in Python because the WebLab server libraries are better intended for this language.

Finally, the client will be developed using \acrshort{html} 5, JavaScript (using JQuery library) and
Bootstrap for a rapid development. It will use the WebLab JavaScript library to communicate with the
experiment server. This library is an asynchronous AJAX library that provides a simple interface to
interact with experiments. We can see an example in algorithm~\ref{alg:weblab_lib}.

\begin{center}
\begin{minipage}{.6\textwidth}
\singlespace
\begin{pyglist}[language=javascript, caption={WebLab JavaScript library example.},
	label={alg:weblab_lib}, listingname={Algorithm}, numbers=left]
// Callback registration that will be
// called after reserving the experiment
Weblab.setOnStartInteractionCallback(start);

// Sending command to the experiment server
Weblab.sendCommand("L", function(response) {
    console.log("Good response: " + response);
}, function(response) {
    console.log("Bad response: " + response);
});
\end{pyglist}
\end{minipage}
\end{center}

\subsection{Deployment Considerations}

TODO: Rasp, bluetooth, wifi, cammeras, weblabtest...

\subsection{Testing Plan}

\subsection{User Manual}

\subsection{Issue Management}
