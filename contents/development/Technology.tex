\section{Tools}

\subsection{Code editor}

For developing this project, some tools have been used. We will analyze them here and see why they
have been used.

The first tool used has been the code editor. The editor used has been Sublime
Text~\cite{sublime_web}. We were using version 2 until March 2015, where development in version 3
gave life signs and we decided to upgrade the version. There are multiple \acrshort{ide}s such as
PyCharm~\cite{pycharm_web}, Eclipse~\cite{eclipse_web} or other editors such as
Atom~\cite{atom_web}, Lime Text~\cite{lime_web} or Microsoft Visual Studio Code~\cite{ms_code_web}.
There are also other console tools like Vim~\cite{vim_web}, Nano~\cite{nano_web} or
Emacs~\cite{emacs_web}.

Nevertheless, even if I would have preferred to work with an open source editor, the truth is that
currently, the closed sourced ones offer better performance. For instance, Atom tries to be the
open source alternative to Sublime Text, but it starts really slowly and the plug-ins for Sublime
Text can be found for almost every situation. Lime Text might outperform Sublime Text in the future
but currently it is still in an unstable release. Microsoft Visual Studio Code was released later
this year and it did not have any benefit with respect to Sublime Text, so it was worthless changing
the editor at this point.

On the other hand, \acrshort{ide}s are big softwares that from my point of view interfere in the
programming process. In the end, this is about feeling comfortable with the tool, and the velocity
and simplicity offered by editors outperform \acrshort{ide}s greatly. That is why the tool used for
all the programming (except the Arduino script) has been Sublime Text 3 (figure~\ref{fig:sublime}).

\begin{figure}[!htbp]
	\centering
	\includegraphics[width=0.3\textwidth]{fig/sublime}
	\caption{Sublime Text logo.}
	\label{fig:sublime}
\end{figure}

Nevertheless, since the robot's microcontroller is an Arduino, the Arduino
\acrshort{ide}~\cite{arduino_web} (figure~\ref{fig:arduino}) has been used to program the robot itself,
since it is simple and easy to use for this simple task.

\begin{figure}[!htbp]
	\centering
	\includegraphics[width=0.6\textwidth]{fig/arduino}
	\caption{Arduino Community logo. \emph{Source: Arduino}}
	\label{fig:arduino}
\end{figure}

\subsection{User interface testing}

TODO: Chrome (+ developer tools), Firefox

\subsection{Server configuration}

TODO: SSH/console

\section{Technology}

In this section we will analyze the technology and tools used for this project.




TODO: Raspberry, Python, HTML 5, JavaScript, JSON, HTTP, REST, Bluetooth,
