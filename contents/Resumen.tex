\chapter*{Resumen}

En respuesta a la fuerte y creciente demanda de nuevas herramientas de aprendizaje, se presenta una
nueva experiencia de aprendizaje, basada en laboratorios remotos. Este proyecto trata de crear una
plataforma de juego para jóvenes de entre 10 y 18 años para introducirlos a la ciencia, tecnología,
ingeniería y las matemáticas de una manera amena.

La plataforma de juego se dividirá en dos escenarios principales. El primero de ellos será un juego
de tipo trivial en el que el usuario podrá controlar el robot en un laberinto y responder algunas
preguntas que darán puntos al usuario. Los ganadores recibirán un premio. Además, en este escenario,
se hará un experimento psicológico con intención de ayudar en la lucha contra la pseudociencia.

El segundo escenario será un servicio en el que el usuario podrá programar el robot desde una
perspectiva visual. Después de eso, el robot deberá realizar algunas tareas simples en el mismo
laberinto del primer escenario.

\vspace{2em}

{\Large\bfseries\sffamily Descriptores}
\vspace{3\medskipamount}

Laboratorios remotos, WebLab-Deusto, Serious Games, Psicología, Pseudociencia
